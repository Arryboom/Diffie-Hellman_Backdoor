\documentclass[a4paper,11pt]{article}

% fonts
\usepackage[utf8]{inputenc}
%\usepackage[francais]{babel}

% to get hyphenation on accented words
\usepackage[T1]{fontenc}

% href
\usepackage{hyperref}
\hypersetup{
    colorlinks=true,
    linkcolor=blue,
    filecolor=blue,      
    urlcolor=blue,
    bookmarks=true
}

% quotes
\usepackage{csquotes}

% code highlighting
\usepackage{minted}
\usemintedstyle{pastie}

% asm
\usepackage{amsmath}
\usepackage{amssymb}
\usepackage{amsthm}

% inline code
\usepackage{listings}
\usepackage{xcolor}

% tables
\usepackage{booktabs}

% algorithm
\usepackage[]{algorithm2e}

% for right cases
\newenvironment{rcases}
  {\left.\begin{aligned}}
  {\end{aligned}\right\rbrace}
  
% images
\usepackage{graphicx}
\usepackage{float}

% diagrams
\usepackage{tikz}
\usetikzlibrary{matrix}
\usetikzlibrary{positioning}

% tables
\usepackage{booktabs}

% no identation
\setlength{\parindent}{0pt}

% theorem
\newtheorem{definition}{Definition}
\newtheorem{property}{Property}
\newtheorem{theorem}{Theorem}

% header
\title{How to backdoor Diffie-Hellman, lessons learned from the Socat non-prime prime}
\author{David Wong}
\date{\emph{NCC Group}, \small{February 2016}}

% 
\begin{document}

\maketitle

\renewcommand{\abstractname}{Abstract}
\begin{abstract}
abstract.\\
\\
\textbf{Keywords:} Socat, backdoor, diffie-hellman, factorization, small subgroup attack, Pohlig-Hellman, Pollard's p-1, ECM\\

\end{abstract}

\section{Introduction}\label{introduction}

Around Christmas 2015, a company called \emph{Juniper} released an out of cycle security bulletin. Two vulnerabilities were being half-disclosed without much details to grasp the seriousness of the situation. Fortunately, at this period of the year, many researchers were home with nothing else to do but solving this puzzle. Quickly, by diffing both the patched and vulnerable binaries, the two issues were pinpointed. While one of the vulnerability was a simple super password implemented at a crucial step of the product's authentication, the other vulnerability discovered was a bit more subtle: a value was modified. More accurately, a number was replaced. The introduction of the vulnerability was so trivial that the simple use of the unix commandline tool \mintinline{bash}{strings} was enough to discover the change.

\begin{figure}[H]
\centering
\includegraphics[scale=.6]{screenOS.png}
\caption{The Juniper's Security Bulletin}\label{screenOS}
\end{figure}

Behind that number was hiding a \textbf{Dual EC} value. Dual EC is a Pseudo-Random Number Generator (PRNG) that is believed to have been backdoored by the NSA \ref{dualec}. The PRNG's core has the ability to provide a Nobody But Us (NOBUS) trapdoor, a backdoor that can only be used by the people holding certain secret values, in our case the elliptic curve discrete logarithm $k$ in the Dual EC equation $Q = [k]P$.

The quest to find the Juniper's backdoor and the following open questions are a fascinating work by itself \ref{juniper}, although it is only the introduction of this work that aims to show how secure and strong cryptographic constructions are a simple and subtle change away from being your own secretive pipe-show.

More recently on February the 1st 2016, \emph{Socat} \href{http://www.openwall.com/lists/oss-security/2016/02/01/4}{published a security advisory of its own}

\begin{displayquote}
In the OpenSSL address implementation the hard coded 1024 bit DH p parameter was not prime. The effective cryptographic strength of a key exchange using these parameters was weaker than the one one could get by using a prime p. Moreover, since there is no indication of how these parameters were chosen, the existence of a trapdoor that makes possible for an eavesdropper to recover the shared secret from a key exchange that uses them cannot be ruled out.\\
A new prime modulus p parameter has been generated by Socat developer using OpenSSL dhparam command.\\
In addition the new parameter is 2048 bit long
\end{displayquote}

This is alerting: in the same vein of Juniper's problem a single number was changed. But Socat's problems root deeper. A year before that suspicious 1024 bit Diffie-Hellman (DH) value was introduced, the free software would serve its TLS connection using a 512 bit DH modulus. The problems of such small DH parameters are already well known \ref{logjam}, the problems of implementing DH securely are unfortunately rarely well understood. The defense approach is told in several RFCs \ref{rfc2785}, but no paper so far take the point of view of the attacker.

RFC on how to implement Diffie-Hellamn: https://tools.ietf.org/html/rfc2631#section-2.4

there exist documentation on how to secure diffie-hellman (rfc rfc rfc) Interestingly there are nothing in litterature from the attacker point of view.

\section{Human error}\label{mistake}

How likely was it a mistake?

-> explain how we usually generate prime
-> how we test for prime
    - probable
    - provable
    ( maybe from here: https://www.ietf.org/rfc/rfc2631.txt )

* Ways we verified that it could have been a mistake

* Ways the primality test could have been wrong -> show that you have more chance winning the superball blabla (check HN thread for that)

-- note --

from hn:


a) knew it was non-prime, and used it to weaken the crypto

b) knew it was non-prime, and used it because they didn't think it needed to be prime (which is a massive sin of ignorance)

c) grabbed 1024 bits of rand() and didn't check if it was prime (again, stupid)

d) grabbed some rand and checked the prime-ness using a bad method

e) used a "prime number generator" that produced bad output


\section{A prime modulus}

while DH can work on any kind of subgroup, the security of composite modulus subgroup are usually not as high as their prime counterparts. The primary reasons are smalls subgroup attack and Pohlig Hellman.

But first, prime or not prime, let's see one problem: what happens if the order of the group created has a "smooth" order (we will see it doesn't have to be smooth per se)

Def. Smooth:

\subsection{CRT}

CRT is a magic cryptanalyis tool. Like lattices for smaller values than the normal, SAT, GNFS, ...

graph to explain how it works

\subsection{Small subgroup attacks}

http://citeseerx.ist.psu.edu/viewdoc/download?rep=rep1&type=pdf&doi=10.1.1.44.5296

(d'un autre lien)> The small subgroup attack was first pointed out by Vanstone[26]; see also van Oorschot and Wiener [36], Anderson and Vaudenay [1], and Lim and Lee [22].

if the order is a composite of small subgroups

why not getting the discrete log in different subgroup and recombine it with CRT? That sounds easier that getting the discrete log in the whole group

\subsection{Pohlig Hellman}

> Static Diffie-Hellman keys are vulnerable to a small subgroup attack

The previous attack is "impractical" because DH is usually used as DHE (ephemeral keys). And since it is an active attack where we need to do several "handshake" (is this called a handshake?)

Pohlig Hellman is a passive attack, and spying on only one conversation allows you to decrypt the rest

same idea, but instead we will just use one public key and reduce it to different subgroups with a magic operation

note: in theory we can't reduce if we're in a smaller group, we can't "escape", we can only go in small subgroups

\section{How to implement a backdoor in DH}

we could use what we saw previously as such:

generate a prime such that the order of the generated group is p-1 a composites where the dlog is "do-able" in its subgroups. We don't need smooth. Doable is defined by your the computation power of the backdoor creator

But then anyone is able to get the order of the group (p-1) and can then factor that order. [[dlog in k bits is factor in 2 kbits no? Check complexities of these]]

It is believed by some that the NSA is a bit more classy than that and would create what we call "Nobody But Us" types of backdoor. That is nobody can reverse them. This was the case of Dual EC (although Dual EC suffered from slowness and bit biases even without the backdoor)

A good way of doing that is to use a non-prime modulus, this way the order would be $\phi(n)$ which is not p-1 anymore (give the formula)

fact: (the one from handbook of applied crypto)
the dlog in Z_n)* reduces to the combination of factorization of n and dlog in Z_p)* for each p factor of n

so people who would want to reverse the backdoor would first have to factor n to guess the order.

\section{Methods to create a backdoor} title)}

\subsection{method 1}

creating a backdoor method 1: p = p_1 p_2 where p_i - 1 smooth

\subsection{Method 2}

\subsection{Method 3}

\section{Methods to reverse the backdoor}

\subsection{trial division}

we found 271 and...

\subsection{pollard p-1}

such a construction as method1 allows for p-1 pollard to work.

\subsubsection{explanation}

explanation of pollard p-1

\subsubsection{results}

We tried it on a proof of concept and: RESULT

we then tried it on socat's dh1024_p and: RESULT

\subsection{ECM}

not tied to the modulus size, but tied to the smallest factor

on our proof of concet: RESULT

on socat's dh1024_p: RESULT

\section{Testing the security of the new modulus}

it was done p = 2q + 1 with q prime

secure

\section{How to secure a DH implementation}

- if the p = 2q + 1 is not done like that, there is a RFC that tells you how to secure such DH

- openssl dhparam

- also some people believe generation of prime is too difficult and that it shouldn't be possible (rfc with predefined dh groups). But then weakdh (or was it logjam rather), everybody used the same hardcoded dh prime, everybody could have/got owned

\section{Conclusion}

- implementing DH correctly is not that hard
- easy to backdoor
- is it really a backdoor? Since we can't factor it... maybe not (maybe give estimations to factor it, and to use the backdoor if it is indeed a backdoor with such big factors)
- people need to verify open source

\newpage

\section*{Acknowledgements}


\newpage

\begin{thebibliography}{1}

\bibitem{dualEC} Bernstein, Lange and Niederhagen {\em \href{https://eprint.iacr.org/2015/767.pdf}{Dual EC: A Standardized Back Door}}

\bibitem{juniper} Juniper {\em Juniper}

\bibitem{logjam} Adrian, Bhargavan, Durumeric, Gaudry, Green, Halderman, Heninger, Springall, Thomé, Valenta,  VanderSloot, Wustrow, Zanella-Béguelin, Zimmermann \em{\href{https://weakdh.org/imperfect-forward-secrecy-ccs15.pdf}{Imperfect Forward Secrecy: How Diffie-Hellman Fails in Practice}}

\bibitem{rfc2785} RFC 2785: \em{\href{https://tools.ietf.org/html/rfc2785}{Methods for Avoiding the "Small-Subgroup" Attacks on the Diffie-Hellman Key Agreement Method for S/MIME}}

\end{thebibliography}

\end{document}